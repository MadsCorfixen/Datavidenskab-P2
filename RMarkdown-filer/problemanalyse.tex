\documentclass[]{article}
\usepackage{lmodern}
\usepackage{amssymb,amsmath}
\usepackage{ifxetex,ifluatex}
\usepackage{fixltx2e} % provides \textsubscript
\ifnum 0\ifxetex 1\fi\ifluatex 1\fi=0 % if pdftex
  \usepackage[T1]{fontenc}
  \usepackage[utf8]{inputenc}
\else % if luatex or xelatex
  \ifxetex
    \usepackage{mathspec}
  \else
    \usepackage{fontspec}
  \fi
  \defaultfontfeatures{Ligatures=TeX,Scale=MatchLowercase}
\fi
% use upquote if available, for straight quotes in verbatim environments
\IfFileExists{upquote.sty}{\usepackage{upquote}}{}
% use microtype if available
\IfFileExists{microtype.sty}{%
\usepackage{microtype}
\UseMicrotypeSet[protrusion]{basicmath} % disable protrusion for tt fonts
}{}
\usepackage[margin=1in]{geometry}
\usepackage{hyperref}
\hypersetup{unicode=true,
            pdftitle={P2 Rapport},
            pdfauthor={Gruppe B2-19},
            pdfborder={0 0 0},
            breaklinks=true}
\urlstyle{same}  % don't use monospace font for urls
\usepackage{color}
\usepackage{fancyvrb}
\newcommand{\VerbBar}{|}
\newcommand{\VERB}{\Verb[commandchars=\\\{\}]}
\DefineVerbatimEnvironment{Highlighting}{Verbatim}{commandchars=\\\{\}}
% Add ',fontsize=\small' for more characters per line
\usepackage{framed}
\definecolor{shadecolor}{RGB}{248,248,248}
\newenvironment{Shaded}{\begin{snugshade}}{\end{snugshade}}
\newcommand{\AlertTok}[1]{\textcolor[rgb]{0.94,0.16,0.16}{#1}}
\newcommand{\AnnotationTok}[1]{\textcolor[rgb]{0.56,0.35,0.01}{\textbf{\textit{#1}}}}
\newcommand{\AttributeTok}[1]{\textcolor[rgb]{0.77,0.63,0.00}{#1}}
\newcommand{\BaseNTok}[1]{\textcolor[rgb]{0.00,0.00,0.81}{#1}}
\newcommand{\BuiltInTok}[1]{#1}
\newcommand{\CharTok}[1]{\textcolor[rgb]{0.31,0.60,0.02}{#1}}
\newcommand{\CommentTok}[1]{\textcolor[rgb]{0.56,0.35,0.01}{\textit{#1}}}
\newcommand{\CommentVarTok}[1]{\textcolor[rgb]{0.56,0.35,0.01}{\textbf{\textit{#1}}}}
\newcommand{\ConstantTok}[1]{\textcolor[rgb]{0.00,0.00,0.00}{#1}}
\newcommand{\ControlFlowTok}[1]{\textcolor[rgb]{0.13,0.29,0.53}{\textbf{#1}}}
\newcommand{\DataTypeTok}[1]{\textcolor[rgb]{0.13,0.29,0.53}{#1}}
\newcommand{\DecValTok}[1]{\textcolor[rgb]{0.00,0.00,0.81}{#1}}
\newcommand{\DocumentationTok}[1]{\textcolor[rgb]{0.56,0.35,0.01}{\textbf{\textit{#1}}}}
\newcommand{\ErrorTok}[1]{\textcolor[rgb]{0.64,0.00,0.00}{\textbf{#1}}}
\newcommand{\ExtensionTok}[1]{#1}
\newcommand{\FloatTok}[1]{\textcolor[rgb]{0.00,0.00,0.81}{#1}}
\newcommand{\FunctionTok}[1]{\textcolor[rgb]{0.00,0.00,0.00}{#1}}
\newcommand{\ImportTok}[1]{#1}
\newcommand{\InformationTok}[1]{\textcolor[rgb]{0.56,0.35,0.01}{\textbf{\textit{#1}}}}
\newcommand{\KeywordTok}[1]{\textcolor[rgb]{0.13,0.29,0.53}{\textbf{#1}}}
\newcommand{\NormalTok}[1]{#1}
\newcommand{\OperatorTok}[1]{\textcolor[rgb]{0.81,0.36,0.00}{\textbf{#1}}}
\newcommand{\OtherTok}[1]{\textcolor[rgb]{0.56,0.35,0.01}{#1}}
\newcommand{\PreprocessorTok}[1]{\textcolor[rgb]{0.56,0.35,0.01}{\textit{#1}}}
\newcommand{\RegionMarkerTok}[1]{#1}
\newcommand{\SpecialCharTok}[1]{\textcolor[rgb]{0.00,0.00,0.00}{#1}}
\newcommand{\SpecialStringTok}[1]{\textcolor[rgb]{0.31,0.60,0.02}{#1}}
\newcommand{\StringTok}[1]{\textcolor[rgb]{0.31,0.60,0.02}{#1}}
\newcommand{\VariableTok}[1]{\textcolor[rgb]{0.00,0.00,0.00}{#1}}
\newcommand{\VerbatimStringTok}[1]{\textcolor[rgb]{0.31,0.60,0.02}{#1}}
\newcommand{\WarningTok}[1]{\textcolor[rgb]{0.56,0.35,0.01}{\textbf{\textit{#1}}}}
\usepackage{graphicx,grffile}
\makeatletter
\def\maxwidth{\ifdim\Gin@nat@width>\linewidth\linewidth\else\Gin@nat@width\fi}
\def\maxheight{\ifdim\Gin@nat@height>\textheight\textheight\else\Gin@nat@height\fi}
\makeatother
% Scale images if necessary, so that they will not overflow the page
% margins by default, and it is still possible to overwrite the defaults
% using explicit options in \includegraphics[width, height, ...]{}
\setkeys{Gin}{width=\maxwidth,height=\maxheight,keepaspectratio}
\IfFileExists{parskip.sty}{%
\usepackage{parskip}
}{% else
\setlength{\parindent}{0pt}
\setlength{\parskip}{6pt plus 2pt minus 1pt}
}
\setlength{\emergencystretch}{3em}  % prevent overfull lines
\providecommand{\tightlist}{%
  \setlength{\itemsep}{0pt}\setlength{\parskip}{0pt}}
\setcounter{secnumdepth}{0}
% Redefines (sub)paragraphs to behave more like sections
\ifx\paragraph\undefined\else
\let\oldparagraph\paragraph
\renewcommand{\paragraph}[1]{\oldparagraph{#1}\mbox{}}
\fi
\ifx\subparagraph\undefined\else
\let\oldsubparagraph\subparagraph
\renewcommand{\subparagraph}[1]{\oldsubparagraph{#1}\mbox{}}
\fi

%%% Use protect on footnotes to avoid problems with footnotes in titles
\let\rmarkdownfootnote\footnote%
\def\footnote{\protect\rmarkdownfootnote}

%%% Change title format to be more compact
\usepackage{titling}

% Create subtitle command for use in maketitle
\providecommand{\subtitle}[1]{
  \posttitle{
    \begin{center}\large#1\end{center}
    }
}

\setlength{\droptitle}{-2em}

  \title{P2 Rapport}
    \pretitle{\vspace{\droptitle}\centering\huge}
  \posttitle{\par}
    \author{Gruppe B2-19}
    \preauthor{\centering\large\emph}
  \postauthor{\par}
      \predate{\centering\large\emph}
  \postdate{\par}
    \date{18/02/2020}


\begin{document}
\maketitle

\begin{Shaded}
\begin{Highlighting}[]
\KeywordTok{library}\NormalTok{(bookdown)}
\end{Highlighting}
\end{Shaded}

\hypertarget{outliers}{%
\subsection{Outliers}\label{outliers}}

En outlier kan beskrives som en observation, der afviger betydeligt fra
resten af den stikprøve, observationen stammer fra (WIKI). Denne
beskrivelse uddybes neden for, hvor to forskellige metoder til at
identificere outliers gennemgås.

Den første metode til at identificere outliers tager udgangpunkt i
fordelingen af observationerne. Hvis fordelingen er tilnærmelsesvis
klokkeformet, gælder følgende figur, hvor \(\mu\) er middelværdien og
\(\sigma\) er standardafvigelse.

?INDSÆT FIGUR HOW? ! {[}Alt text{]}
(/Users/madsc/OneDrive/Documents/GitHun/Datavidenskab-P2/RMarkdown-filer/Figurer/Normalfordeling.PNG)

Det gælder altså, at \(\approx 68 \%\) af observationerne findes inden
for en standardafvigelse fra middelværdien, \(\approx 95 \%\) inden for
to standardafvigelser og \(\approx 99,7 \%\) inden for tre
standardafvigelser.

På baggrund af dette, kan en outlier defineres som en værdi, der ligger
et vist antal standardafvigelser væk fra middelværdien. Som udgangspunkt
siges det, at hvis en værdi er mere end \(3\) standardafvigelser væk fra
middelværdien, kan den ses som en outlier.

En anden måde at beskrive fordelingen af en observation er ved hjælp af
percentiler - et punkt for hvilket en hvis mængde af data antager
værdier over eller under. Her en medianen et specialtilfælde af en
percentil, 50-percentilen, hvor halvdelen af data antager værdier under
punktet, og den anden halvdel over punktet. Generelt defineres en
percentil som\ldots{}

\BeginKnitrBlock{definition}

\protect\hypertarget{def:unnamed-chunk-2}{}{(\#def:unnamed-chunk-2)
}Test

\EndKnitrBlock{definition}

\emph{p-percentilen} er et punkt, således at \(p \%\) af observationerne
antager værdier under punktet, og \((100 - p) \%\) af observationerne
antager værdier over punktet.

To andre særlige tilfælde af percentiler er 25-percentilen, den nedre
kvartil (\(Q_L\)), og 75-percentilen, den øvre kvartil \(Q_U\). Ved
hjælp af \(Q_L\) og \(Q_U\) kan variabiliteten af observationerne
beskrives som afstanden mellem de to kvartiler. Denne afstand benævnes
\emph{Inner Quartile Range}, IQR, og beregnes som \(IQR = Q_U - Q_L\).

Når både minimum, \(Q_L\), median, \(Q_U\) og maksimum er fundet, kan
disse benyttes til at tegne et boksplot af observationerne. Et boksplot
giver et grafisk overblik af midten og variabiliteten af
observationerne. Et boksplot ser ud, som på nedenstående figur\ldots{}

?INDSÆT FIGUR HOW?

På baggrund af IQR og boksplottet kan outliers nu defineres som data,
der antager værdier mere end \(1,5(IQR)\) gange over \(Q_U\) eller under
\(Q_L\).

Outliers er altså værdier, der er ekstreme nok til at påvirke en
dataanalyse i forkert retning. Derfor benyttes forskellige metoder til
at identificere sådanne outliers, som derefter kan fjernes i en
rengøring af datasættet.

(Medmindre andet er angivet, er kilden vores ASTA-bog)


\end{document}
